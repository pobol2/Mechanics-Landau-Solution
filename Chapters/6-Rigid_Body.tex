\chapter{Motion of a Rigid Body}

% ============ Problem 1 ============ %
\begin{problem}
{
Determine the principle moments of inertia for the followingg types of molecule, regarded as systems of particles at fixed distances apart:
}{}{}
\begin{subproblem}
{
a molecule of collinear atoms 
}
{
Choose the $x_3$-axis along the line containing all atoms, and place the origin at the centre of mass.  Then each particle has coordinates $(0,0,x_a)$, so the inertia tensor is diagonal in the basis
$(x_1,x_2,x_3)$, with
\begin{equation*}
    I_3 = 0, \qquad I_1 = I_2.
\end{equation*}
Indeed, since the distance to the $x_1$-axis (or $x_2$-axis) is $|x_a-x_{CM}|$ for each particle,
\begin{equation*}
I_1 = I_2 = \sum_a m_a\,(x_a-x_{CM})^2,
\qquad
x_{CM}=\frac{1}{\mu}\sum_a m_a x_a,\quad \mu=\sum_a m_a.
\end{equation*}
Expanding and using the definition of $x_{CM}$ gives
\begin{align*}
I_1=I_2
&= \sum_a m_a x_a^2 - 2x_{CM}\sum_a m_a x_a + \mu x_{CM}^2 \\
&= \sum_a m_a x_a^2 - \frac{1}{\mu}\left(\sum_a m_a x_a\right)^2.
\end{align*}
To rewrite this in terms of pairwise separations, expand the square:
\begin{equation*}
\left(\sum_a m_a x_a\right)^2
= \sum_a m_a^2 x_a^2 + 2\sum_{a<b} m_a m_b x_a x_b.
\end{equation*}
Hence
\begin{align*}
I_1=I_2
&= \sum_a m_a x_a^2
-\frac{1}{\mu}\sum_a m_a^2 x_a^2
-\frac{2}{\mu}\sum_{a<b} m_a m_b x_a x_b \\
&= \frac{1}{\mu}\sum_a m_a(\mu-m_a)x_a^2
-\frac{2}{\mu}\sum_{a<b} m_a m_b x_a x_b.
\end{align*}
Now note that $\mu-m_a=\sum_{b\neq a}m_b$, so
\begin{equation*}
\sum_a m_a(\mu-m_a)x_a^2
=\sum_a\sum_{b\neq a} m_a m_b x_a^2
=\sum_{a<b} m_a m_b(x_a^2+x_b^2).
\end{equation*}
Therefore
\begin{align*}
I_1=I_2
&= \frac{1}{\mu}\sum_{a<b} m_a m_b(x_a^2+x_b^2)
-\frac{2}{\mu}\sum_{a<b} m_a m_b x_a x_b \\
&= \frac{1}{\mu}\sum_{a<b} m_a m_b (x_a-x_b)^2.
\end{align*}
Since for collinear atoms the pair distance is $l_{ab}=|x_a-x_b|$, we finally obtain
}
{
\begin{align*}
    I_1&=I_2=\frac{1}{\mu}\sum_{a<b} m_a m_b\, l_{ab}^2 \\
    I_3&=0
\end{align*}
}
\end{subproblem}
\begin{subproblem}
{
a triatomic molecule which is an isosceles triangle (Fig. 36)
}
{
Let the molecule lie in the $(x_1,x_2)$-plane, with $x_1$ along the base and $x_2$ along the axis of symmetry. The two base atoms have equal masses $m_1$ and are separated by a distance $a$; the apex atom has mass $m_2$ and is at height $h$ above the base midpoint. Denote the total mass by
\begin{equation*}
    \mu = 2m_1 + m_2.
\end{equation*}
Choose coordinates (before shifting to the centre of mass):
\begin{equation*}
( x_1,x_2 )_{(1)}=\left(-\frac a2,\,0\right),\qquad
( x_1,x_2 )_{(3)}=\left(+\frac a2,\,0\right),\qquad
( x_1,x_2 )_{(2)}=\left(0,\,h\right),
\end{equation*}
where atoms $(1)$ and $(3)$ have mass $m_1$ and atom $(2)$ has mass $m_2$. By symmetry $x_{1,CM}=0$, and
\begin{equation*}
x_{2,CM}=\frac{m_2 h}{\mu}.
\end{equation*}
The symmetry implies that the axes through the centre of mass parallel to $x_1$, $x_2$, and $x_3$ are principal. Hence,
\begin{equation*}
I_1 = \sum_a m_a (x_{2,a}-x_{2,CM})^2,\qquad
I_2 = \sum_a m_a (x_{1,a}-x_{1,CM})^2,\qquad
I_3 = I_1+I_2,
\end{equation*}
since the molecule is planar ($x_3\equiv 0$). Using the pair-distance identity in one dimension,
\begin{equation*}
\sum_a m_a (x_{2,a}-x_{2,CM})^2=\frac1\mu\sum_{a<b} m_a m_b (x_{2,a}-x_{2,b})^2,
\end{equation*}
and noting that the only non-zero vertical separations are between the apex and each base atom, equal to $h$, we obtain
\begin{equation*}
I_1=\frac1\mu\Big( m_1m_2 h^2 + m_1m_2 h^2 \Big)
=\frac{2m_1m_2}{\mu}\,h^2.
\end{equation*}
Similarly,
\begin{equation*}
I_2=\sum_a m_a (x_{1,a}-x_{1,CM})^2
=\frac1\mu\sum_{a<b} m_a m_b (x_{1,a}-x_{1,b})^2.
\end{equation*}
The horizontal separations are between the two base atoms, and between the apex and either base atom. Hence
\begin{equation*}
I_2=\frac1\mu\left(m_1^2a^2 + 2m_1m_2\left(\frac a2\right)^2\right)
=\frac1\mu\left(m_1^2a^2+\frac12 m_1m_2a^2\right)
=\frac12 m_1 a^2.
\end{equation*}
Therefore
\begin{equation*}
I_3 = I_1+I_2 = \frac{2m_1m_2}{\mu}\,h^2 + \frac12 m_1 a^2.
\end{equation*}
}
{
\begin{align*}
    I_1&=\frac{2m_1m_2}{\mu}\,h^2 \\
    I_2&=\frac12 m_1 a^2 \\
    I_3&=I_1+I_2.
\end{align*}
}
\end{subproblem}
\begin{subproblem}
{
a tetratomic molecule which is an equilateral-based tetrahedron
}
{
The base is an equilateral triangle of side $a$ in the plane $x_3=0$, with three equal masses $m_1$ at its vertices. A fourth atom of mass $m_2$ lies on the symmetry axis through the base centroid at height $h$ above the base. Let
\begin{equation*}
\mu = 3m_1 + m_2.
\end{equation*}
Let $\rho$ be the distance from the base centroid to a base vertex. For an equilateral triangle,
\begin{equation*}
\rho=\frac{a}{\sqrt3},\qquad \rho^2=\frac{a^2}{3}.
\end{equation*}
By symmetry the centre of mass lies on the symmetry axis. Measuring $x_3$ from the base plane,
\begin{equation*}
x_{3,CM}=\frac{m_2 h}{\mu}.
\end{equation*}
The symmetry axis is a principal axis; any two orthogonal axes through the centre of mass perpendicular to it are also principal and have equal moments. Thus
\begin{equation*}
I_1=I_2,\qquad I_3 \text{ about the symmetry axis } (x_3).
\end{equation*}
The apex mass lies on the axis and contributes nothing. Each base mass is at distance $\rho$ from the axis, hence
\begin{equation*}
I_3 = 3m_1\rho^2 = 3m_1\frac{a^2}{3} = m_1 a^2.
\end{equation*}
Let $x_1$ be any axis through the centre of mass perpendicular to $x_3$. Then
\begin{equation*}
I_1=\sum_a m_a\,(y_a^2+z_a^2),
\end{equation*}
where $y$ is a coordinate orthogonal to $x_1$ in the plane perpendicular to the axis, and $z$ is measured along $x_3$
from the centre of mass. The vertical coordinates relative to the centre of mass are
\begin{equation*}
z_{\text{base}}=-x_{3,CM}=-\frac{m_2h}{\mu},\qquad
z_{\text{apex}}=h-x_{3,CM}=h\left(1-\frac{m_2}{\mu}\right)=\frac{3m_1}{\mu}h.
\end{equation*}
For the three vertices of an equilateral triangle, for any in-plane axis through the centroid,
\begin{equation*}
\sum_{\text{base}} y^2 = \frac{3}{2}\rho^2 = \frac{3}{2}\cdot\frac{a^2}{3}=\frac12 a^2.
\end{equation*}
Therefore,
\begin{align*}
I_1
&= m_1\left(\sum_{\text{base}} y^2\right) + 3m_1 z_{\text{base}}^2 + m_2 z_{\text{apex}}^2 \\
&= \frac12 m_1 a^2 + 3m_1\left(\frac{m_2^2 h^2}{\mu^2}\right) + m_2\left(\frac{9m_1^2 h^2}{\mu^2}\right) \\
&= \frac12 m_1 a^2 + \frac{3m_1m_2}{\mu}\,h^2.
\end{align*}
Hence $I_2=I_1$.
}
{
\begin{align*}
    I_1&=I_2=\frac{3m_1m_2}{\mu}\,h^2+\frac12 m_1 a^2 \\
    I_3&=m_1 a^2
\end{align*}
}
\end{subproblem}