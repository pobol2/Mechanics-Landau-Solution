\chapter{Collisions Between Particles}

% ============ Problem 1 ============ %

\begin{problem}
{
Find the relation between the angles $\theta_1$, $\theta_2$ (in the $L$ system) after disintegrating into two particles.
}
{
The angles of the particles, in the $C$ system, are related by
\begin{equation*}
\theta_0 = \theta_{10} = \pi - \theta_{20},
\end{equation*}
where $\theta_0$ as been defined to simplify the notation. From the formulae (16.5) of the book, we have
\begin{align*}
    V + v_{10}\cos{\theta_0} &= v_{10} \sin{\theta_0} \cot{\theta_1} \\
    V - v_{20}\cos{\theta_0} &= v_{20} \sin{\theta_0} \cot{\theta_2} .
\end{align*}
We must eliminate $\theta_0$ from the two equations above. To do so, we can first solve for $\cos{\theta_0}$ and $\sin{\theta_0}$ which give us
\begin{align}
    &\sin{\theta_0} = \frac{V + v_{10}\cos{\theta_0}}{v_{10}\cot{\theta_1}} = \frac{V - v_{20}\cos{\theta_0}}{v_{20}\cot{\theta_2}} \label{C4P1_sin} \\
    \Rightarrow& Vv_{20}\cot{\theta_2} + v_{10}v_{20}\cos{\theta_0}\cot{\theta_2} = Vv_{10}\cot{\theta_1} - v_{10}v_{20}\cos{\theta_0}\cot{\theta_1} \nonumber \\
    \Rightarrow& \cos{\theta_0} = \frac{V\left( v_{10}\cot{\theta_1} - v_{20}\cot{\theta_2} \right)}{v_{10}v_{20} \left( \cot{\theta_1} + \cot{\theta_2} \right)}.  \label{C4P1_cos}
\end{align}
Using the sum of square of \eqref{C4P1_sin} and \eqref{C4P1_cos}, we get
\begin{align*}
    &\left( \frac{V + v_{10}\cos{\theta_0}}{v_{10}\cot{\theta_1}} \right)^2 + \left( \frac{V\left( v_{10}\cot{\theta_1} - v_{20}\cot{\theta_2} \right)}{v_{10}v_{20} \left( \cot{\theta_1} + \cot{\theta_2} \right)} \right)^2 = 1 \\
    \Rightarrow& \left( v_{10} + v_{20} \right)^2 + \left( v_{10}\cot{\theta_1} - v_{20}\cot{\theta_2} \right)^2 = \frac{v_{10}^2v_{20}^2}{V^2} \left( \cot{\theta_1} + \cot{\theta_2} \right)^2 \\
    \Rightarrow& v_{10}^2 \csc{\theta_1}^2 + v_{20}^2 \csc{\theta_2}^2 - 2 v_{10}v_{20}\cot{\theta_1}\cot{\theta_2} = \frac{v_{10}^2v_{20}^2\sin^2{(\theta_1+\theta_2)}}{V^2\sin^2{\theta_1}\sin^2{\theta_2}} \\ 
    \Rightarrow& v_{20}^2 \sin^2{\theta_1} + v_{10}^2 \sin^2{\theta_2} - 2 v_{10}v_{20}\sin{\theta_1}\sin{\theta_2}\cos{\theta_1}\cos{\theta_2} = \frac{v_{10}^2v_{20}^2}{V^2}\sin^2{(\theta_1+\theta_2)}.
\end{align*}
Using the equation (16.2) from the book and the relation $v_{10}/v_{20} = m_2/m_1$, we obtain
}
{
\begin{equation*}
    \frac{m_2}{m_1} \sin^2{\theta_2} + \frac{m_1}{m_2} \sin^2{\theta_1} - 2 \sin{\theta_1}\sin{\theta_2}\cos{\theta_1}\cos{\theta_2} = \frac{2\epsilon}{(m_1+m_2)V^2} \sin^2{(\theta_1+\theta_2)}.
\end{equation*}
}
\end{problem}

% ============ Problem 2 ============ %

\begin{problem}
{
    Find the angular distribution of the resulting particles in the $L$ system.
}
{
The distribution of the resulting particles with respect to the angle $\theta_0$ is given by the equation (16.7) of the book, \ie
\begin{equation*}
    \frac{1}{2}\sin{\theta_0}d\theta_0 = \frac{1}{2}\mathrm{d}(\cos{\theta_0}).
\end{equation*}
Using equation (16.6) of the book yield
\begin{equation*}
    \frac{1}{2}\mathrm{d}(\cos{\theta_0}) = -\frac{V}{v_0}\sin{\theta}\cos{\theta}\mathrm{d}\theta \pm \left( \sin{\theta}\sqrt{1-(V^2/v_0^2)\sin^2{\theta}} + \cos{\theta}\frac{(V^2/v_0^2)\sin{\theta}\cos{\theta}}{\sqrt{1-(V^2/v_0^2)\sin^2{\theta}}} \right) \mathrm{d}\theta.
\end{equation*}
After simplification, the angular distribution become
}
{
\begin{equation*}
    \frac{1}{2}\sin{\theta} \mathrm{d}\theta \left( 2\frac{V}{v_0}\cos{\theta} \pm \frac{1+(V^2/v_0^2)\cos{2\theta}}{\sqrt{1-(V^2/v_0^2)\sin^2{\theta}}} \right)
\end{equation*}
}
\end{problem}

% ============ Problem 3 ============ %
% See https://www.physicsforums.com/threads/trigonometric-function-mechanics-landau.962697/

\begin{problem}
{
Determine the range of possible values of the angle $\theta$ between the directions of motion of the two resulting particles in the $L$ system.
}
{
The tangent of the anglle $\theta = \theta_1 + \theta_2$ is
\begin{equation*}
    \tan{\theta} = \frac{\tan{\theta_1} + \tan{\theta_2}}{1 - \tan{\theta_1}\tan{\theta_2}}.
\end{equation*}
Using equation (16.5) of the book to get $\tan{\theta_1}$ and $\tan{\theta_1}$, and using $\theta_0 = \theta_{10} = \pi - \theta_{20}$, the last equation become
\begin{equation*}
    \tan{\theta} = \frac{(v_{10}+v_{20})V\sin{\theta_0}}{V^2+(v_{10}-v_{20})V\cos{\theta_0} - v_{10}v_{20}}
\end{equation*}
or equivalently
\begin{equation*}
    \cot{\theta} = \frac{V^2+(v_{10}-v_{20})V\cos{\theta_0} - v_{10}v_{20}}{(v_{10}+v_{20})V\sin{\theta_0}}.
\end{equation*}
In order to find the extrema, we take the derivative with respect to $\theta_0$,
\begin{align*}
    &\pd{\cot{\theta}}{\theta_0} = 0 = \frac{(-v_{10}+v_{20})V^2 - V\cos{\theta_0}(V^2-v_{10}v_{20})}{(v_{10}+v_{20})V^2\sin^2{\theta_0}}\\
    \Rightarrow &\cos{\theta_0} = \frac{(v_{20}-v_{10})V}{V^2-v_{10}v_{20}}.
\end{align*}
We know that $-1 \leq \cos{\theta_0} \leq 1$, thus $v_{10} < V < v_{20}$. Moreover, we could calculate the derivative of the cotangent with respect to $ V $ and find the expression:
\begin{equation*}
\frac{\partial \cot{\theta}}{\partial V} = \frac{V^2 + v_{10}v_{20}}{(v_{10}+v_{20})V^2 \sin{\theta_0}}
\end{equation*}
which is positive for all values of $ V $.

This is why $ \cot{\theta} $ may assume all the values between 0 and $ \pi $ when $ v_{10} < V < v_{20} $, because $ \frac{\partial \cot{\theta}}{\partial \theta_0} \neq 0 $ and $ \frac{\partial \cot{\theta}}{\partial V} \neq 0 $ for all $ V, \theta_0 $ in that interval. It is easy to see that if $ V $ is kept constant and $ \theta_0 $ is varied between 0 and $ \pi $, $ \cot{\theta} $ spans the interval $ (-\infty, +\infty) $; similarly, if $ \theta_0 $ is kept constant and $ V $ is varied between 0 and $ +\infty $, $ \cot{\theta} $ spans the interval $ (-\infty, +\infty) $.

When $ V \leq v_{10} $, it is easy to see that $ \cot{\theta} $ has a maximum value. In fact,
\begin{equation*}
    \lim_{\theta_0 \to 0^+} \cot(x) = -\infty, \quad \lim_{\theta_0 \to \pi^-} \cot(x) = -\infty
\end{equation*}
and the maximum value of the cotangent is reached when:
\begin{equation*}
    \cos{\theta_0} = \frac{(v_{20} - v_{10})V}{V^2 - v_{10}v_{20}}
\end{equation*}
The calculation of the maximum value leads to:
\begin{equation*}
    \cot{\theta} = \frac{V^2 + (v_{10} - v_{20})V\cos{\theta_0} - v_{10}v_{20}}{(v_{10}+v_{20})V \sin{\theta_0}} = - \frac{\sqrt{(V^2 - v_{10}v_{20})^2 - (v_{20} - v_{10})^2 V^2}}{(v_{10} + v_{20})V}
\end{equation*}
Then, after transforming the cotangent into a sine, one gets:
\begin{equation*}
    \sin{\theta} = \frac{(v_{20} + v_{10})V}{V^2 + v_{10}v_{20}}
\end{equation*}
If we put:
\begin{equation*}
    \alpha = \sin^{-1}\left( \frac{(v_{20} + v_{10})V}{V^2 + v_{10}v_{20}} \right)
\end{equation*}
then, in the interval where $ V \leq v_{10} $, we have the following ranges for $ \theta $: $ \pi - \alpha \leq \theta \leq \pi $.
Similarly, when $ V \geq v_{20} $, it is easy to see that $ \cot{\theta} $ has a minimum value. In fact,
\begin{equation*}
    \lim_{\theta_0 \to 0^+} \cot(x) = +\infty, \quad \lim_{\theta_0 \to \pi^-} \cot(x) = +\infty
\end{equation*}
and the minimum value of the cotangent is reached when $ \theta = \alpha $. In this case, the ranges for $ \theta $ are $ 0 \leq \theta \leq \alpha $ (the equality $ \theta = 0 $ is satisfied only when $ V \to +\infty $). The final solution is thus
}
{
\begin{align*}
    &\text{if } v_{10} < V < v_{20} \quad \Rightarrow \quad 0 < \theta < \pi \\
    &\text{if } V \leq v_{10} \quad \Rightarrow \quad \pi - \alpha \leq \theta \leq \pi \\
    &\text{if } V \geq v_{20} \quad \Rightarrow \quad 0 \leq \theta \leq \alpha
\end{align*}
where
\begin{equation*}
\alpha = \sin^{-1} \left( \frac{(v_{20} + v_{10})V}{V^2 + v_{10}v_{20}} \right).
\end{equation*}
}
\end{problem}